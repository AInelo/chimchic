%--------------A VOTRE ATTENTION-------------%
% Les étudiants en master qui disposent de plus de 3 chapitres dans leurs travaux peuvent en complèter
% Les Membres doivent figurer dans la dernière version finale du mémoire après soutenance pour dépôt de mémoire

\documentclass{ifri}
\usepackage{titletoc}
\setlength{\glsdescwidth}{0.65\textwidth}
% \usepackage{lscape}

\typeMemoire{Diplôme de Licence en Informatique}
\optionFormation{Système d'Information et Réseau Informatique}
\etudiant{Lionel \textbf{TOTON}}
\titreDuMemoire{Création d'une plateforme web pour la réalisation des plans d'affaires sur mesure au Bénin.} %Implémention pour une meilleure sécurité dans les réseau LAN sous IPv6 // Proposition: Identification des vulnérabilités dans un reseau LAN IPv6 et mesures pour une meilleure sécurité.

\dateSoutenance{/2024}
%\promo{2\up{ème}}
\anneeScolaire{2022-2023}


%%maitre de mémoire
\encadrants{Pierre Jerôme \textbf{ZOHOU}}

%% Membres du Jury
\jurys{%
\begin{tabular}{llll}
	Nom et prénoms du président & Grade & Entité & Président \\
	Nom et prénoms de l'examinateur & Grade & Entité & Examinateur \\
	Nom et prénoms du rapporteur & Grade & Entité & Rapporteur \\
\end{tabular}	
}


\hypersetup{
 pdftitle={--},
 pdfauthor={--},
 pdfsubject={--},
 pdfkeywords={--} 
 }

\color{bookColor}

%importation du glossaire
\loadglsentries{glossaire_reduit}

\begin{document}

\pageDeGarde
%\pageTitre

\pagecolor{white}

%% page vide
%\thispagestyle{empty}\ \clearpage


\selectlanguage{french}

% sommaire
\pagenumbering{roman}

\setcounter{tocdepth}{0}
\startlist{toc}
\printlist{toc}{}{\chapter*{Sommaire}}
\setcounter{tocdepth}{5}

%% rdedicaces
\include{0-dedicace}
\newpage 

%% remerciements
\include{1-remerciements}
\newpage 

% Résume
\include{2-resume}
\newpage

%liste des figures
\listoffigures 
\newpage

%liste des tableaux
\listoftables
\newpage

%liste des algo
%\selectlanguage{french}
%\listofalgorithmes
%\newpage

% Les sigles et acronymes
\setglossarystyle{altlist}
\printglossary[title=Liste des acronymess, toctitle=Liste des acronymes, type=\acronymtype]

\newpage

% Le glossaire proprement dit
%\setglossarystyle{super}
%\printglossary[type=main]


\pagenumbering{arabic}
\setcounter{page}{1}
%%introduction
\include{3-introduction}
%\lhead[]{} \rhead[]{} \chead[]{}
\selectlanguage{french}
\fancyhead[L]{\tiny \leftmark}
\fancyhead[R]{\scriptsize \rightmark}
\fancyfoot[C]{\thepage}

\chapter{Revu de littérature}\label{chap:1}
 \input{1-partie/1-fichier}
 
 \chapter{Analyse, conception et choix technique}\label{chap:2}
 \input{2-partie/1-fichier}
 
\chapter{Présentation du prototype de l’application
et discussion}\label{chap:3}
 \input{3-partie/1-fichier}
 
% \include{perspectives}
%%conclusion
\include{4-conclusion}
% 
\lhead[]{} \rhead[]{} \chead[]{}

%%biblio
\addcontentsline{toc}{chapter}{Bibliographie}
\bibliographystyle{abbrv}
\bibliography{biblio}


%\include{annexe}

\newpage
\tableofcontents

\end{document}          
